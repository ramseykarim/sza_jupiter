\documentclass{article}

\usepackage[margin=1in]{geometry}
\usepackage{amsmath,graphicx}

\newcommand{\av}[1]{\langle #1 \rangle}

\title{Averaging Methods Investigation}
\author{Ramsey Karim}
\date{\today}

\begin{document}

\maketitle

\section*{Background}
We are given two values for each point: flux and error.
Flux is stated to be in units of janskys and is denoted here as $F$.
Error, denoted here as $\rho$, is presumed to be given as a fraction such that
$\rho F = \sigma$ where sigma is the error in units of janksys.

\section{} \label{s:1}
The first method of flux averagine shown here was never implemented;
it's more of a test of the data.
The flux values are the standard deviations.
The error is averaged assuming thermal error only, which is what we had done initially.
$$ \frac{1}{\av{\sigma}^{2}} = \sum_{i} \frac{1}{\sigma_{i}^{2}} $$

\section{} \label{s:2}
This method of flux averaging is the only one we have used.
This method of error averaging was corrected from the previous and was used
for the poster, the modelling, and the points I emailed to Dave and Imke.
$$ \av{F} = \frac{\sum \frac{F_{i}}{\sigma_{i}}}{\sum \frac{1}{\sigma_{i}}} $$
$$ \frac{1}{\av{\sigma}^{2}} =  \frac{\sum \frac{1}{\sigma_{i}} (F_{i} - \av{F})^{2} }{\sum \frac{1}{\sigma_{i}}} $$
This may not have been accurate to what I had in my write up, which I did not realize until now.
I believe I was getting strange error values until I decided to divide by the weight of the sums. \\
The main issue here is that the units are a little odd.
The flux averaging comes out in the correct units, but through an odd venue.
Recall that $\rho F = \sigma$.
This means that
$$ \av{F} = \frac{\sum \frac{F_{i}}{\rho_{i} F_{i}}}{\sum \frac{1}{\rho_{i} F_{i}}} $$
which simplifies to
$$ \av{F} = \frac{\sum \frac{1}{\rho_{i}}}{\sum \frac{1}{\rho_{i} F_{i}}} $$
which has the correct units but doesn't intuitively suggest ``average flux''
(to me). \\
The average error has incorrect units.
Perhaps if we placed the $(F_{i} - \av{F})^{2}$ on the bottom of the expression instead of the top,
in which case the units would be correct and the error would increase if the standard deviation of
that point were higher.

\section{} \label{s:3}
This is the version I left our last meeting intending to implement, according to my notes.
It looks slightly different from the previous and yields rather small error bars, which
may indicate some sort of an issue here too. \\
Flux averaging is the same as the last one.
Error is as:
$$ \frac{1}{\av{\sigma}^{2}} = \sum \frac{1}{\sigma_{i}^{2}} (F_{i} - \av{F})^{2} $$
I don't think this was what we intended to leave it as,
and I may have written it down incorrectly.
The units here don't work either; the average error turns unitless.

\section{} \label{s:4}
Regarding the flux averaging, I believe that this formulation must be correct,
but I have little to no intuition as to why. \\
Regarding the error averaging, I have two suggestions.
\subsection{} \label{s:41}
First of all, we could modify the formula in Method \ref{s:2} so that the $(F_{i} - \av{F})^{2}$
term is on the bottom.
In addition, the $\sigma_{i}$ terms seem like they should be squared.
$$ \av{\sigma}^{2} =  \frac{\sum \frac{1}{(F_{i} - \av{F})^{2} \sigma_{i}^{2}}}{\sum \frac{1}{\sigma_{i}^{2}}} $$
This yields the correct units and a better relationship between error and deviation from the mean.
The problem here is that the errors are being used as weights, which doesn't seem to be a good way
to focus on averaging the weights.
\subsection{} \label{s:42}
My second suggestion tries to adhere more closely to the formula in Method \ref{s:1}.
Here I try summing up the inverse errors again, but with the addition of modifying them by the
standard deviation as a fraction of the average.
$$ \frac{1}{\av{\sigma}^{2}} = \sum \frac{1}{\sigma_{i}^{2} (F_{i} - \av{F})^{2} } $$
This formula preserves correct units ($\sigma$ in janskys),
retains the proper relationship between error and devation from mean,
and stays truest to the original method of averaging error. \\

\end{document}